%%%%%%%%%%%%%%%%%%%%%%%
%% Author: Conghao Wong
%% Date: 2021-07-19 19:19:13
%% LastEditors: Conghao Wong
%% LastEditTime: 2021-07-19 20:56:58
%% Description: file content
%% Github: https://github.com/conghaowoooong
%% Copyright 2021 Conghao Wong, All Rights Reserved.
%%%%%%%%%%%%%%%%%%%%%%%

\documentclass[../paper.tex]{subfiles}

\begin{document}
    
\section{Experiments}

\subsection{Experimental Setup}

\subsubsection{Datasets}
We evaluate \MODEL~on three public available trajectory datasets, ETH\cite{youWillNeverWalkAlone}, UCY\cite{2007Crowds}, and Standford Drone Dataset (SDD)\cite{learningSocialEtiquette}.
They are consist of different agents' trajectories with rich social interactions and scene constraints in various scenarios.

(a) \textbf{The ETH-UCY Benchmark}:
Many previous methods \cite{socialLSTM,socialGAN,sophie} take several sub-datasets from ETH and UCY, i.e., eth and hotel from the ETH dataset, plus univ, zara1, zara2 from the UCY dataset, to train or evaluate their models with the ``leave-one-out'' strategy.
The ETH-UCY benchmark contains 1536 pedestrians in total with thousands of non-linear trajectories.
The annotations are their coordinates in meters.

(b) \textbf{Standford Drone Dataset}:
the SDD\cite{learningSocialEtiquette} is a popular computer vision dataset.
It canbe used in object dection, tracking, trajectory prediction, and many other tasks.
It has 60 bird-view videos captured by drones over the Standford campus.
More than 11,000 different kinds of agents are annotated in bounding boxes.
They have over 185,000 social behaviors, and 40,000 scene interactive behaviors.


\subsubsection{Metrics}
We employ two metrics to evaluating the prediction accuracy, including the Mean Average Displacement (MAD) error and the Final Average Displacement (FAD) error \cite{youWillNeverWalkAlone}.

(a) \textbf{MAD}:
Mean Euclidean distance between ground truth and predicted points of all steps.
Formally,
\begin{equation}
    \mbox{MAD}(y, \hat{y}) = \frac{1}{t_f} \sum_{t} \Vert p_t - \hat{p}_t \Vert_2.
\end{equation}

(b) \textbf{FAD}:
Euclidean distance between the last point's prediction and ground truth.
Formally,
\begin{equation}
    \mbox{FAD}(y, \hat{y}) = \Vert p_{t_h + t_f} - \hat{p}_{t_h + t_f} \Vert_2.
\end{equation}

\subsubsection{Baselines}
We choose several state-of-the-art methods across both deterministic and generative models as our baselines, they are:

\textbf{\SRLSTM}:

\textbf{\TRANSFORMER}:

\textbf{\SOCIALGAN}:

\textbf{\SOPHIE}:

\textbf{\BIGAT}:

\textbf{\PEEKING}:

\textbf{\SIMAUG}:

\subsubsection{Implementation Details}

\subsection{Comparision to State-of-the-Art Methods}

\subsection{Discusses}

\subsubsection{Quantitative Analysis}

\subsubsection{Qualitative Analysis}

\subsubsection{Failure Cases}

\subsection{Visualization}


\section{Conclusion}


\end{document}